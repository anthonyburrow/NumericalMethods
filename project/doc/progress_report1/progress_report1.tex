\documentclass[12pt]{article}

\usepackage[utf8]{inputenc}
\usepackage[margin=1in]{geometry}
\renewcommand{\baselinestretch}{1}
\usepackage{indentfirst}

\usepackage{amsmath, amssymb}

\usepackage{hyperref}
\usepackage{cleveref}
\usepackage{graphicx}
\usepackage{float}
\graphicspath{{./figs/}}

\usepackage{natbib}
\bibliographystyle{aasjournal}

\begin{document}

\begin{center}\begin{LARGE}
\textbf{Final Project Progress Report \#1}
\end{LARGE}\end{center}


\section*{}

Our project is to develop a 1D hydrocode that functions to model a high-mass
($10\ M_\odot$) star that has undergone a core-collapse event and subsequent
shock that ejects its outermost layers. We are effectively following the
numerical method given in \citet{arnett66} to calculate temperatures,
pressures, densities, etc. using equations of mass conservation, momentum
conservation, and energy conservation, as well as an equation of state.

So far, we have set up the code to handle most of the hydrodynamical
functionality. This means writing all the difference equations for all of the
mechanical properties (zone velocities, specific volumes, etc.) as well as some
others (e.g. temperature).

Some functionalities still needed are those that are related to the
equation of state (pressure and energy). This is a bit more difficult to decide
on due to the nature of the project, but after looking through and
understanding the paper more, we may assume that the pressure can be split into
that due to nucleons (as free Fermi particles), electrons (relativistic Fermi
particles), and radiation. Using the pressure and energy equations of state
from \citet{arnett66}, most of what is left in the code is to actually
calculate the pressure and evalutate $\partial E / \partial T$ and
$\partial E / \partial V$, which is still a bit unclear for now. If this proves
too time-consuming or difficult, more simplistic equations of state could also
be used (e.g. just the relativistic degenerate Fermi gas).

There are also the radiative terms to calculate, such as energy generation
rate ($\dot{s}$ due to neutrino emission, etc.) and opacities. This will
require a lot more looking into.

After these have all been established, boundary conditions must be set. The
boundary conditions we still need to apply are the initial temperature,
pressure, and density profiles. These profiles will possibly just be some
power-law relation with an initial core value.

Ultimately, the project is going well in terms of writing the code. However, on
my end I am a bit stuck with some of the physical mechanisms involved in a
core-collapse event, and what I should actually be including in the code.
Unfortunately I do not yet have plots to show, however to make things more
clear when writing this code, my next step will probably be to set it up to run
and actually generate plots, regardless of what the plots look like.

\vspace{5mm}

Our current progress may be found at the following GitHub link:
\url{https://github.com/anthonyburrow/Hydro1D}

\bibliography{progress_report1}

\end{document}
