\documentclass[12pt]{article}

\usepackage[utf8]{inputenc}
\usepackage[margin=1in]{geometry}
\renewcommand{\baselinestretch}{1.2}
\usepackage{indentfirst}

\usepackage{amsmath, amssymb}

\begin{document}

\begin{center}\begin{LARGE}
\textbf{Assignment 1: Results/Description}
\end{LARGE}\end{center}

\section*{Problem 1}

I've created a program \texttt{myquadratic_single} in C++ that solves a given
quadratic equation. The coefficients \texttt{a}, \texttt{b}, and \texttt{c}
corresponding to the equation are provided by the user in a parameter file. The
program puts the equation into the form $x^2 + 2bx + c = 0$. It then performs
the calculation of the solution in the standard way (i.e. with addition and
subraction of the discriminant term), as well as a more precise method that
makes use of division in place of subtraction (subtraction is dangerous).

I tested it with a couple known cases. First, I tried the trivial equation
$x^2 - 1 = 0$ (\texttt{a=1}, \texttt{b=0}, \texttt{c=-1}), and the output of
the program is as expected:
\begin{verbatim}
Reading from parameter file: ./config/params
Attempting to solve equation 1x^2 + 0x + -1 = 0
Standard calculation:
-1  1
Accurate calculation:
-1  1
\end{verbatim}

I also tested the equation $3.2x^2 + x - 6.5 = 0$, and the solutions are given to
\texttt{double} precision:
\begin{verbatim}
Reading from parameter file: ./config/params
Attempting to solve equation 3.2x^2 + 1x + -6.5 = 0
Standard calculation:
-1.5900087183693077  1.2775087183693077
Accurate calculation:
-1.5900087183693077  1.2775087183693079
\end{verbatim}

\section*{Problem 2}

I've placed the quadratic solver in its own implementation/header ("quadratic.cpp/hpp"),
and one may \texttt{include} this functionality as a subroutine, etc. in another
file with an \texttt{\#include "quadratic.hpp"}. This has been done in the second main
implementation "myquadratic.cpp".

\section*{Problem 3}

I have written the Makefile for this program, which compiles each object and
links them to create the executable file. To compile this program, one may just
run \texttt{make}. This creates the "myquadratic" executable, and it does the
exact same thing as before, but it allows more flexibility in writing the code.

\section*{Problem 4}

\subsection*{(a)}

As a test for imaginary roots, I can just use the equation $x^2 + 1 = 0$. When
doing this, in C++ the \texttt{std::sqrt} function returns \texttt{NaN}, so no
result is found. However, an error is not given and the code continues to run,
which could be dangerous or non-informative.

One may alter the code to handle these imaginary solutions in a few ways. The
simplistic approach that I chose for this simple program was to incorporate an
exception to be thrown if an imaginary solution arises. The output from my
program to such an equation provided gives:
\begin{verbatim}
Reading from parameter file: ./config/params
Attempting to solve equation 1x^2 + 0x + 1 = 0
Error: Imaginary value(s) encountered in solution.
\end{verbatim}

One may also incorporate complex number functionality if available.
\texttt{complex} objects are available in \texttt{std::complex} for C++, for
example. These of course require over double the amount of memory, though.

\subsection*{(b)}

With the extreme equation $x^2 - 200000x + 1 = 0$, my program outputs the
following:
\begin{verbatim}
Reading from parameter file: ./config/params
Attempting to solve equation 1x^2 + -200000x + 1 = 0
Standard calculation:
4.9999944167211652e-06  199999.99999500002
Accurate calculation:
199999.99999500002  5.0000000001249996e-06
\end{verbatim}
One of the solutions is slightly different from the other (albeit still close
to somewhat high precision). As was mentioned, the standard calculation uses
subtraction, which has the potential to destroy some significant precision at
the cost of faster processing time. On the other hand, the accurate method only
performs a division to get the second solution from the first, leading to a
more precise yet slower result.

\end{document}
